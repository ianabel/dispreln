
%
% Copyright Ian G. Abel 2018
% 

\documentclass[prb,aps,amssymb,amsmath,a4paper]{revtex4}
\usepackage{graphicx}
\usepackage{bm,ulem,color}

\sloppy


\newcommand{\collop}[1][f_s]{\ensuremath{{C}\hspace{-0.5mm}\left[#1\right]}}
\newcommand{\vth}[1][s]{\ensuremath{v_{\mathrm{th}_#1}}}
\newcommand{\vt}{\ensuremath{v_{\mathrm{th}}}}
\newcommand{\pd}[2]{\ensuremath{ \frac{\partial #1} {\partial #2} } }
\newcommand{\inpd}[2]{\ensuremath{ \infrac{\partial #1} {\partial #2} } }
\newcommand{\gyror}[1]{\ensuremath{ {\left< #1 \right>}_{\bm{r}}}}
\newcommand{\ensav}[1]{\ensuremath{ {\left< #1 \right>}_{\mathrm{turb}}}}  % Ensemble averaging
\newcommand{\fav}[1]{\ensuremath{\left< #1 \right>_{\psi}}} % Flux surface averaging
\newcommand{\gyroR}[1]{\ensuremath{{\left< #1 \right>}_{\bm{R}}}}

\providecommand{\eqref}[1]{Eq.\ (\ref{#1})}
\newcommand{\eqsref}[2]{Eqs.\ (\ref{#1}) and (\ref{#2})}
\newcommand{\eqsdash}[2]{Eqs.\ (\ref{#1})--(\ref{#2})}
\newcommand{\Secref}[1]{Section\ \ref{#1}}
\newcommand{\secref}[1]{Sec.\ \ref{#1}}
\newcommand{\apref}[1]{Appendix\ \ref{#1}}
\providecommand{\Apref}[1]{Appendix\ \ref{#1}}
\newcommand{\Figref}[1]{Fig.\ \ref{#1}}
\providecommand{\Tref}[1]{Table.\ \ref{#1}}

\providecommand{\Or}[1]{\mathcal{O}#1}
\providecommand{\Eref}[1]{Equation\ (\ref{#1})}
\providecommand{\eref}[1]{(\ref{#1})}


\newcommand{\uv}[1]{\bm{\hat{#1}}} % uv = unit vector
\newcommand{\dg}{\cdot\nabla}      % dg = dot grad
\providecommand{\dv}{\nabla\cdot}
\providecommand{\curl}{\nabla\times}
\newcommand{\pb}[2]{\left\{ {#1} \middle., {#2} \right\}} % pb = poisson bracket

% Change the following commands to alter the notation of the document.

\newcommand{\tor}{\phi} % toroidal angle
\newcommand{\gyr}{\vartheta} % gyroangle
\newcommand{\pot}{\varphi} % electrostatic potential
\newcommand{\fpot}{\Phi} % Flow potential
\newcommand{\pol}{\theta} % Poloidal angle (straight field)
\newcommand{\gkeps}{\epsilon} % Rho over L, gyrokinetic epsilon
\newcommand{\aspect}{\epsilon} % Inverse aspect ratio
\newcommand{\energy}[1][s]{{{\varepsilon}_#1}} % Energy Variable
\newcommand{\denergy}[1][s]{{\dot{\varepsilon}}_{#1}}
\newcommand{\torflux}{\Psi} % Toroidal flux
\newcommand{\source}[1][s]{\ensuremath{{{{S}}_{#1}}}} % General Source
\newcommand{\gkupot}{\delta \pot'}
\newcommand{\gkpot}{\chi}
\newcommand{\psistar}[1][s]{{\psi^*_#1}}
\newcommand{\psistardot}[1][s]{{\dot{\psi}^*_#1}}
\newcommand{\magmom}[1][s]{\mu_#1}
\newcommand{\dmu}[1][s]{\dot{\mu}_#1}
\newcommand{\jacob}{\mathcal{J}}
%\newcommand{\idmat}{\bm{\underline{\mathbb{I}}}} % Identity Matrix
\newcommand{\tu}{\hat{u}_{\parallel s}}

\newcommand{\ddR}[2][s]{\pd{#2}{\bm{R}_{#1}}}
\newcommand{\dgR}[2][s]{\cdot\ddR[#1]{#2}}

\newcommand{\tcite}[1]{Ref.\ \onlinecite{#1}}
\providecommand{\tensor}[1]{ {\bm{\mathsf{#1}}}}
\newcommand{\viscosity}[1][s]{\tensor{\Pi}_#1}
\newcommand{\idmat}{\tensor{I}}
\newcommand{\ParticleFlux}[1][s]{\Gamma_{#1}}
\newcommand{\HeatFlux}[1][s]{q_{#1}}
\newcommand{\MomentumFlux}[1][s]{\pi^{(\psi\tor)}_{#1}}
\newcommand{\TotMomFlux}{{\pi}^{(\psi\tor)}_{\mathrm{tot}}}
\newcommand{\inertia}{J}
\newcommand{\accel}[1][s]{\bm{a}_{#1}}
\newcommand{\daccel}[1][s]{\delta \bm{a}_{#1}}
% partial with respect to time at constant psi
\newcommand{\ddtpsi}{\left.\pd{}{t}\right|_{\psi}}
\newcommand{\maryland}{
Department of Physics, University of Maryland, College Park, MD 20742-4111, USA
}
\newcommand{\chalmers}{
Department of Physics, Chalmers University of Technology, G\"oteborg, SE-41296, Sweden
}

\newcommand{\wint}{\int\hspace{-1.25mm} d^3 \bm{w}}
\newcommand{\Fneo}[1][s]{F^{\mathrm{(nc)}}_{#1}}
\newcommand{\FE}{F^{\mathrm{(E)}}_{s}}
\newcommand{\Fnct}{\tilde{F}^{\mathrm{(nc)}}_{s}}
\newcommand{\Fstar}[1][s]{F^{*}_{1#1}}
\newcommand{\Esource}[1][s]{S^{({E})}_{#1}}
\newcommand{\Psource}[1][s]{S^{({n})}_{#1}}
\newcommand{\Msource}{S^{({\omega})}}
\newcommand{\FreeEnergy}{W}
\newcommand{\perpav}[1]{\left<#1\right>_\perp}
\newcommand{\timeav}[1]{\left<#1\right>_T}
\newcommand{\injpow}{P_{\mathrm{inj}}}
\newcommand{\entropy}[1][]{\widetilde{H}_{#1}}
\newcommand{\MeanEntropy}[1][]{H_{#1}}
\newcommand{\deltaEntropy}[1][]{\delta H_{#1}}
\newcommand{\entropyflux}{\bm{\Gamma}^{({H})}}
\newcommand{\infrac}[2]{ \left.{#1}\middle/{#2}\right. }
\newcommand{\binfrac}[2]{\left(\infrac{#1}{#2}\right)}
\newcommand{\CollEnergy}[1][s]{C^{{(E)}}_#1}
\newcommand{\ViscousHeat}[1][s]{P^{\mathrm{visc}}_#1}
\newcommand{\JouleHeat}[1][s]{P^{\mathrm{Ohm}}_#1}
\newcommand{\PotEng}[1][s]{P^{\mathrm{pot}}_#1}
\newcommand{\CompHeat}[1][s]{P^{\mathrm{comp}}_#1}
\newcommand{\TurbInj}[1][s]{P^{\mathrm{drive}}_#1}
\newcommand{\TurbColl}[1][s]{P^{\mathrm{diss}}_#1}
\newcommand{\TurbPow}[1][s]{P^{\mathrm{turb}}_#1}
\newcommand{\EMViscosity}{\pi_{\mathrm{EM}}^{(\psi\tor)}}
\newcommand{\NeoMomFlux}[1][s]{\pi^{\mathrm{(nc)}}_{#1}}
\newcommand{\ClassMomFlux}[1][s]{\pi^{\mathrm{(cl)}}_{#1}}
\newcommand{\vchi}{\bm{V}_\chi}
\newcommand{\vchiR}{\gyroR{\vchi}}
\newcommand{\vdrift}[1][s]{\bm{V}_{\mathrm{D}#1}}
\newcommand{\GammaTurb}{\Gamma_{s}^{\mathrm{turb}}}
\newcommand{\QTurb}{q_{s}^{\mathrm{turb}}}
\newcommand{\PiTurb}{\pi_{s}^{\mathrm{turb}}}
\newcommand{\EProd}{C^{\mathrm{(H)}}}
\newcommand{\Rentropyflux}{\Gamma^{({H})}}
\newcommand{\JUflux}{\Gamma^{{(U)}}}
\newcommand{\CollEntropy}{C^{{(H)}}}
\newcommand{\EntropySource}{S^{{(H)}}}
\newcommand{\DebyeLength}{\lambda_{\mathrm{De}}}
\newcommand{\NotN}[1][s]{N_{#1}}
\newcommand{\massratio}{\delta}

\newcommand{\delB}{{\ensuremath{\delta\bm{B}}}}
\newcommand{\delBp}{{\ensuremath{\delta B_\parallel}}}
\newcommand{\delBpN}[1]{{\ensuremath{\delta B_\parallel^{(#1)}}}}
\newcommand{\delAp}{{\ensuremath{\delta A_\parallel}}}
\newcommand{\delApN}[1]{{\ensuremath{\delta A_\parallel^{(#1)}}}}
\newcommand{\delE}{{\ensuremath{\delta\bm{E}}}}
\newcommand{\delA}{{\ensuremath{\delta\bm{A}}}}
\newcommand{\delj}{{\ensuremath{\delta\bm{j}}}}
\newcommand{\delpot}{{\delta\pot}}
\newcommand{\MagDelB}{{\delta B}} \newcommand{\bhat}{{\widetilde{\bm{b}}}}

\newcommand{\MeanB}{{\bm{B}}}
\newcommand{\MeanE}{{\bm{E}}}
\newcommand{\MeanMagB}{{B}}
\newcommand{\Meanb}{{\bm{b}}}
\newcommand{\Meanj}{{\bm{j}}}

\newcommand{\Efield}{{\widetilde{\bm{E}}}}
\newcommand{\Bfield}{{\widetilde{\bm{B}}}}
\newcommand{\MagBfield}{{\widetilde{B}}}

\newcommand{\current}{{\widetilde{\bm{j}}}}
\newcommand{\chargedens}{{\widetilde{\varrho}}}
\newcommand{\Apot}{{\widetilde{\bm{A}}}}
\newcommand{\Epot}{{\widetilde{\pot}}}
\newcommand{\MeanA}{{\bm{A}}}
\newcommand{\MeanPot}{{\pot}}

% Averages and abbreviations for paper 2
\newcommand{\bav}[1]{\left<#1\right>_{\parallel}}
\newcommand{\veff}{\bm{u}_{\mathrm{eff}}}
\newcommand{\fluctfav}[1]{\left<#1\right>_{\tilde\psi}}

\newcommand{\upar}[1][e]{\delta u_{\parallel #1}}
\newcommand{\tpsi}{\tilde{\psi}}
\newcommand{\talpha}{\tilde{\alpha}}
\newcommand{\tl}{\tilde{l}}
\newcommand{\ddttwiddles}[1][]{\left.\pd{#1}{t}\right|_{\tilde{\psi},\tilde{\alpha},\tilde{l}}}
\newcommand{\qpar}{\delta q_{\parallel e}}

\newcommand{\FHat}[1][s]{\widehat{F}_{1#1}}
\newcommand{\angvel}{\omega}
\newcommand{\cycfreq}[1][s]{\Omega_{#1}}

\newcommand{\ExEnergy}[1][s]{\widetilde{\varepsilon}_{#1}}
\newcommand{\ExMagmom}[1][s]{\widetilde{\mu}_{#1}}
\newcommand{\ExGyr}{\widetilde{\gyr}}
\newcommand{\dExGyr}{\dot{\widetilde{\gyr}}}
\newcommand{\dExenergy}[1][s]{{\dot{\widetilde{\varepsilon}}}_{#1}}
\newcommand{\magmomN}[2][s]{{\mu}_{#2#1}}
\newcommand{\mmbar}[2][s]{\overbar{\magmomN[#1]{#2}}}
\newcommand{\dmmbar}[2][s]{\overbar{\delta\magmomN[#1]{#2}}}
\newcommand{\vA}{v_{\mathrm{A}}} % Alfvèn velocity
\newcommand{\cS}{c_{\mathrm{s}}} % Alfvèn velocity
\newcommand{\bpol}{\beta_{\mathrm{pol}}}
\providecommand{\omstar}[1][]{\ensuremath{\omega_{*#1}}}
%\newcommand{\Tref}[1]{Table~\ref{#1}}
%\newcommand{\tref}[1]{table~\ref{#1}}

\newcommand{\chempot}[1][s]{\Upsilon_{#1}}
\newcommand{\quantconc}[1][s]{n_{\mathrm{Q}#1}}

\newcommand{\vpsi}{\bm{V}_\psi}

\newcommand{\tw}{\widetilde{\bm{w}}}
\newcommand{\twint}{\int d^{3}\widetilde{\bm{w}}}
\newcommand{\Magtw}{\widetilde{{w}}}

\newcommand{\dgt}{\cdot\widetilde{\nabla}_\perp}

\newcommand{\MaxVDrift}{\widehat{\bm{V}}_D}
\newcommand{\ddtpsitwiddles}{\left.\pd{ }{t}\right|_{\tpsi}}
\newcommand{\Pturb}{P^{\mathrm{turb}}_e}
\newcommand{\Pmean}{P^{\mathrm{mean}}_e}
\newcommand{\tvp}{\tilde{V}'}
\newcommand{\tee}{\tilde{\varepsilon}_e}
\newcommand{\delNe}{\overbar{\delta n}_e}
\newcommand{\delTe}{\overbar{\delta T}_e}

\newcommand{\delEnt}[1][s]{{\Delta S_{#1}}}

\newcommand{\overbar}[1]{\mkern 1.5mu\overline{\mkern-1.5mu#1\mkern-1.5mu}\mkern 1.5mu}
\newcommand{\nustar}[1][e]{\nu^{*}_{#1}}
\newcommand{\shat}{\hat{s}}

\newcommand{\lincol}[1][\cdot]{\ensuremath{{C}_{L}\hspace{-0.5mm}\left[#1\right]}}
\newcommand{\npol}{\ensuremath{n^{\mathrm{pol}}_s}}
\newcommand{\poldrift}{\ensuremath{\bm{V}^{\mathrm{pol}}_s}}

\newcommand{\ash}{\ensuremath{\mathrm{ash}}}

\newcommand{\rhos}{\ensuremath{\rho_{\mathrm{s}}}} % Sound radius
\newcommand{\cs}{\ensuremath{c_{\mathrm{s}}}} % Sound speed

\newcommand{\gstwo}{{\ttfamily GS2}}
\newcommand{\trinity}{{\ttfamily TRINITY}}

\RequirePackage{amsmath}
\makeatletter
\newcommand*\rel@kern[1]{\kern#1\dimexpr\macc@kerna}
\newcommand*\widebar[1]{%
  \begingroup
  \def\mathaccent##1##2{%
    \rel@kern{0.8}%
    \overline{\rel@kern{-0.8}\macc@nucleus\rel@kern{0.2}}%
    \rel@kern{-0.2}%
  }%
  \macc@depth\@ne
  \let\math@bgroup\@empty \let\math@egroup\macc@set@skewchar
  \mathsurround\z@ \frozen@everymath{\mathgroup\macc@group\relax}%
  \macc@set@skewchar\relax
  \let\mathaccentV\macc@nested@a
  \macc@nested@a\relax111{#1}%
  \endgroup
}
\makeatother

\providecommand{\coll}[4]{ \ensuremath{C_{#1\,#2}[#3,#4]}}
\providecommand{\lorentz}[1]{\ensuremath{\mathcal{L}}[#1]}
\providecommand{\nuD}[2]{\ensuremath{\nu_{D}^{#1\,#2}}}
\providecommand{\nup}[2]{\ensuremath{\nu_{\parallel}^{#1\,#2}}}
\providecommand{\nus}[2]{\ensuremath{\nu_{s}^{#1\,#2}}}
\providecommand{\nu}[2]{\ensuremath{\nu_{#1\,#2}}}
\providecommand{\injeng}{\varepsilon_*}
\providecommand{\injvel}{v_*}
\begin{document}

\date{\today}

\title{The Linear Theory of a Gyrokinetic Slab}

\maketitle


\section{Gyrokinetic Equations in a Slab}


\section{The Linearised Equations}


Linearising (\ref{gk1}) merely results in neglecting the nonlinear interaction term $[h,\gyroR{\chi}]$. The gyrokinetic maxwell equations are already linear in the fluctuating quantities $h$,$\phi$,$A_\parallel$ and $\delta B_\parallel$. The way we will proceed is to use Fourier expansion in both real and gyrocenter space to solve the linearisation of (\ref{gk1}) for $h$ in terms of the fluctuating fields, use this to express the moments of $h$ that appear in Maxwell's equations, then derive a consistency condition on those. In order to expand all quantities as Fourier modes we carefully expand $h_s$ in a Fourier mode with spatial dependence on $\mathbf{R}_s$ and our fields in modes with spatial dependence on $\mathbf{r}$ , consistent with our full equations. We get,
\begin{align}
\nonumber
h_s &= \hat{h}_s e^{i \left( \mathbf{k}\cdot\mathbf{R}_s - \omega t \right)}\\\nonumber 
\phi &= \hat\phi e^{i \left( \mathbf{k}\cdot\mathbf{r}_s - \omega t \right)}\\\nonumber 
A_\parallel &= \widehat{A_\parallel} e^{i \left( \mathbf{k}\cdot\mathbf{r}_s - \omega t \right)}\\\nonumber 
\delta B_\parallel &= \widehat{\delta B}_\parallel e^{i \left( \mathbf{k}\cdot\mathbf{r}_s - \omega t \right)}
\end{align}
We need to derive from these a form for $\gyroR{\chi}$ , thus from the definition we have,
\begin{align}
\nonumber
\gyroR{\chi} &= \frac{1}{2\pi} \int \phi - \frac{1}{c} \mathbf{v}\cdot{A} \,\mathrm{d}\theta\\\nonumber
&= e^{i \left( \mathbf{k}\cdot\mathbf{R}_s - \omega t \right)} \left( \hat{\phi} - \frac{v_\parallel \widehat{A_\parallel}}{c}\right)\left(\frac{1}{2\pi} \int e^{i a_s \cos \theta} \,\mathrm{d}\theta \right) + e^{i \left( \mathbf{k}\cdot\mathbf{R}_s - \omega t \right)}\frac{\widehat{\delta B}_\parallel v_\perp}{c k_\perp} \left(\frac{1}{2\pi} \int \cos\theta e^{i a_s \cos \theta} \,\mathrm{d}\theta\right) \\\nonumber
	&= J_0(a_s) \left( \phi - \frac{v_\parallel A_\parallel}{c} \right) + J_1(a_s) \frac{v_\perp}{k_\perp} \frac{\delta B_\parallel }{c}
\end{align}
Where we have used the integral definitions of the Bessel functions $J_0(a_s)$ and $J_1(a_s)$ and defined $a_s = k_\perp v_\perp / \Omega_s$
	Inserting these into the equations, cancelling the eikonal and dropping hats we obtain,
\begin{align}
\label{gklin}
-i\omega h_s + iv_\parallel k_\parallel h_s = \gyroR{ e^{-i\mathbf{k}\cdot\mathbf{R}_s} \collop[ h ] } + \frac{ic}{B_0}\pd{F_{0s}}{\mathbf{R}} \times \mathbf{k} \cdot \mathbf{\hat{z}} \gyroR{\chi} + i q_s \omega \gyroR{\chi} \pd{F_{0s}}{\mathcal{E}}
\end{align}
At this stage we will assume a Maxwellian background for the systems, 
				\[
				F_{0s} = \frac{n_{0s}}{\pi^{3/2} \vth^3} e^{v^2/\vth^2}
				\]
It will also be useful to express $F_{0s}$ in terms of $T_{0s} = \frac{1}{2} m_s \vth^2$ in order to express the gradients of $F_{0s}$ in terms of $\nabla \ln n_{0s}$ and $\nabla \ln T_{0s}$. Thus
\begin{align}
\nonumber
\nabla F_{0s} = \pd{F_{0s}}{\mathbf{R}} &= \nabla n_{0s} \pd{F_{0s}}{n} + \nabla T_{0s} \pd{F_{0s}}{T} \\\nonumber
&= \left( \nabla \ln n_{0s} + \left( \frac{\mathcal{E}}{T_{0s}} - \frac{3}{2} \right) \nabla \ln T_{0s} \right) F_{0s}\\ \nonumber
&= \left( \bm{\kappa}_{s} + \frac{v^2}{\vth^2} \bm{\eta}_{s} \right) F_{0s}
\end{align}
Where we have defined two dimensionless vectors $\bm{\kappa}$ and $\bm{\eta}_s$ which contain the gradients of $F_{0s}$.
Thus we can solve (\ref{gklin}) for $h$ by neglecting the collisional terms as  many cases of interest are fundamentally collisionless.
\begin{align}
h_s = \frac{F_{0s}}{\omega - v_\parallel k_\parallel } \left(\frac{\omega q_s}{T_{0s}} - \left(\bm{\kappa} + \frac{v^2}{\vth^2} \bm{\eta}_s \right) \times \mathbf{k} \cdot \mathbf{\hat{z}} \right) \left(J_0(a_s) \left( \phi - \frac{v_\parallel A_\parallel}{c} \right) + J_1(a_s) \frac{v_\perp}{k_\perp} \frac{\delta B_\parallel }{c} \right)
\end{align}
The Maxwell equations become 
\begin{align}
\label{gkqnlin}
\sum\limits_{s} - \frac{ q_s \phi}{T_{0s}}{F_{0s}}  + \int J_0(a_s){h_s} \, \mathrm{d}^3\mathbf{v} = 0
\end{align}
For the quasi-neutrality condition and,
\begin{align}
\label{gkampparlin}
- k_\perp^2 A_\parallel &= \frac{4\pi}{c} \sum\limits_s q_s \int v_\parallel J_0(a_s){h_s} \, \mathrm{d}^3\mathbf{v}
\\
\label{gkampperplin}
- k_{\perp}^2 \delta B_\parallel &= \frac{4\pi}{c} \sum\limits_s q_s \int k_\perp v_\perp J_1(a_s) {h_s} \, \mathrm{d}^3\mathbf{v}
\end{align}

For the parallel and perpendicular components of Amp\`ere's Law, where we have taken the divergence of the perpendicular Amp\`ere's Law in order to reduce it to a scalar equation for $\delta B_\parallel$. Thus our problem has been reduced to computing the moments of $h_s$ in these equations, and solving a system of algebraic equations for the fluctuating fields.

\section{Velocity Integrals}
In order to compute the velocity integrals of $h$ derived in the previous section we will need to perform several integrations over $v_\parallel$ and $v_\perp$. Thus we use,
	\[
	\int \mathrm{d}^3\mathbf{v} = \int\limits_0^\infty v_\perp \,\mathrm{d}v_\perp \int\limits_{-\infty}^{\infty} \mathrm{d}v_\parallel \int\limits_0^{2\pi} \mathrm{d}\theta
	\]
	To factorise the integrals into those over parallel and perpendicular velocities. The perpendicular velocity integrals include the integrals over Bessel functions that occur from ring and gyro-averaging, damped at the upper limit of the integral by the Maxwellian factor in $h$. These integrals are performed by use of the identity,
	\begin{align}
	\int\limits_0^\infty J_n(a x)J_n(b x) e^{-c^2x^2} x \,\mathrm{d}x = \frac{1}{2c^2} I_n\left(\frac{ab}{2c^2}\right) e^{-\left(a^2+b^2\right)/4a^2} 
	\end{align}
	From (\cite{watson1966ttb}), partial integration and standard results from (\cite{gradshteuin1994tis}). The specific results obtained we will need are,
	\begin{align}
	\Gamma^{l}_{n\,m} &= 2 \int\limits_0^\infty \left(\frac{k_\perp \rho_s}{2}\right)^{-(n+m)} J_{n}(a_s)J_m(a_s) \left(\frac{v_\perp}{\vth}\right)^l e^{-v_\perp^2/\vth^2} \,\mathrm{d}\left(\frac{v_\perp}{\vth}\right)
	\\
	\Gamma^{1}_{0\,0} &= e^{-\alpha}I_{0}(\alpha)
	\\
	\Gamma^{2}_{1\,0} &= e^{-\alpha} \left( I_0(\alpha)-I_1(\alpha)\right)
	\\
	\Gamma^{3}_{0\,0} &= e^{-\alpha}\left(\alpha\left(I_1(\alpha) - I_0(\alpha)\right) + I_0(\alpha)\right)
	\\
	\Gamma^{4}_{1\,0} &= e^{-\alpha} \left( 2(1-\alpha)I_0(\alpha) - \left(1 - 2\alpha\right)I_1(\alpha)\right)
	\\
	\Gamma^{3}_{1\,1} &= 2 e^{-\alpha} \left(I_0(\alpha) - I_1(\alpha) \right)
	\\
	\Gamma^{5}_{1\,1} &= 2 e^{-\alpha} \left( \left(3-2\alpha\right)I_0(\alpha) - 2\left(1-\alpha\right) I_1(\alpha)\right) 
\end{align}
The parallel velocity integrals all reduce to those of the form
\begin{align}
\zeta_n\left(\frac{\omega}{k_\parallel \vth}\right) &= - \frac{k_\parallel}{\sqrt{\pi}\vth^{n+1}} \int\limits_{-\infty}^\infty \frac{v^n_\parallel e^{-v_\parallel^2/\vth}}{\omega - v_\parallel k_\parallel } \,\mathrm{d}v_\parallel
\end{align}
Performing these integrals requires choosing a definition for them that makes both physical and mathematical sense if $\omega$ is real. This problem was originally encountered in linearising the collisionless Vlasov equation for longitudinal plasma oscillations. The three plausible options were either to perturb the contour of integration above or below the singularity at $v_\parallel = \omega / k_\parallel$, or to take the principle part of the integral. The latter was the method used by Vlasov, and results in no damping of the linear waves. The correct result however was derived by Landau (\cite{landau}) by considering the full initial value problem, to perturb the contour below the singularity. This leads to wave damping as we shall see later and also motivates the following definition,
\begin{align}
Z(\xi) = \frac{1}{\sqrt{\pi}} \int\limits_L \frac{e^{-x^2}}{x - \xi} \,\mathrm{d}x
\end{align}
$Z(\xi)$ is known as the Plasma Dispersion function (\cite{fried1961pdf}) and the integration is over the Landau contour, below the singularity. This function is defined by analytic continuation for all complex values of $\xi$. As analytic limits for this function are readily available we express all the integrals $\zeta_n$ in terms of $Z$ to obtain,
\begin{align}
\zeta_0\left(\xi\right) &= Z\left( \xi\right)
\\
\zeta_1\left(\xi\right) &= \left( 1 + \xi Z(\xi) \right)
\\
\zeta_2\left(\xi\right) &= \xi \left( 1 + \xi Z(\xi) \right)
\\
\zeta_3\left(\xi\right) &= \left( \xi^2 \left(1 + \xi Z(\xi)\right) + \frac{1}{2} \right)
\\
\zeta_4\left(\xi\right) &= \xi  \left( \xi^2 \left(1 + \xi Z(\xi)\right) + \frac{1}{2} \right)
\end{align}
We can now write down the moments of $h_s$ in terms of these integrals.
\begin{align}
\nonumber
\int\gyror{h_s}\,\mathrm{d}^3\mathbf{v}= -& \left(\frac{q_s\omega}{T_{0s}} - \frac{c}{B_0} \bm{\kappa}\times\mathbf{k}\cdot\mathbf{\hat{z}}\right)\left(\frac{n_{0s}}{\vth^3}\right)\left\{ \frac{1}{k_\parallel}\zeta_0\left(\phi \Gamma^{1}_{0\,0} \vth^2 + \frac{\rho_s \delta B_\parallel \vth^3}{2 c} \Gamma^{2}_{1\,0}\right) - \frac{\vth^3 A_\parallel}{k_\parallel c} \zeta_1 \Gamma^{1}_{0\,0} \right\}
\\ \nonumber
+& \frac{n_{0s}}{\vth^3} \frac{c}{B_0} \bm{\eta}_s\times\mathbf{k}\cdot\mathbf{\hat{z}}\left\{ \frac{1}{k_\parallel} \zeta_0 \left( \phi \Gamma^{3}_{0\,0} \vth^2 + \frac{\rho_s \delta B_\parallel \vth^3}{2 c} \Gamma^{4}_{1\,0} \right) - \frac{A_\parallel \vth^3}{k_\parallel c} \zeta_1 \Gamma^{3}_{0\,0}\right. \\
		\nonumber +& \left. \frac{1}{k_\parallel} \zeta_2 \left(\phi \Gamma^{1}_{0\,0} \vth^2 + \frac{\rho_s \delta B_\parallel \vth^3}{2 c} \Gamma^{2}_{1\,0} \right) - \frac{A_\parallel \vth^3}{k_\parallel c} \zeta_3\Gamma^{1}_{0\,0}\right\}
\end{align}

\begin{align}
\nonumber
\int v_\parallel\gyror{h_s}\,\mathrm{d}^3\mathbf{v}= -& \left(\frac{q_s\omega}{T_{0s}} - \frac{c}{B_0} \bm{\kappa}\times\mathbf{k}\cdot\mathbf{\hat{z}}\right)\left(\frac{n_{0s}}{\vth^3}\right)\left\{ \frac{\vth}{k_\parallel}\zeta_1\left(\phi \Gamma^{1}_{0\,0} \vth^2 + \frac{\rho_s \delta B_\parallel \vth^3}{2 c} \Gamma^{2}_{1\,0}\right) - \frac{\vth^4 A_\parallel}{k_\parallel c} \zeta_2 \Gamma^{1}_{0\,0} \right\}
\\ \nonumber
+& \frac{n_{0s}}{\vth^3} \frac{c}{B_0} \bm{\eta}_s\times\mathbf{k}\cdot\mathbf{\hat{z}}\left\{ \frac{\vth}{k_\parallel} \zeta_1 \left( \phi \Gamma^{3}_{0\,0} \vth^2 + \frac{\rho_s \delta B_\parallel \vth^3}{2 c} \Gamma^{4}_{1\,0} \right) - \frac{A_\parallel \vth^4}{k_\parallel c} \zeta_2 \Gamma^{3}_{0\,0}\right. \\
		\nonumber +& \left. \frac{\vth}{k_\parallel} \zeta_3 \left(\phi \Gamma^{1}_{0\,0} \vth^2 + \frac{\rho_s \delta B_\parallel \vth^3}{2 c} \Gamma^{2}_{1\,0} \right) - \frac{A_\parallel \vth^4}{k_\parallel c} \zeta_4\Gamma^{1}_{0\,0}\right\}
\end{align}

\begin{align}
\nonumber
\nabla\cdot\int\gyror{\mathbf{\hat{z}}\times\mathbf{v}_\perp h_s}\,\mathrm{d}^3\mathbf{v}= -& \left(\frac{q_s\omega}{T_{0s}} - \frac{c}{B_0} \bm{\kappa}\times\mathbf{k}\cdot\mathbf{\hat{z}}\right)\left(\frac{k_\perp \rho_s n_{0s}}{2}\right)\left\{ \frac{1}{k_\parallel}\zeta_0\left(\phi \Gamma^{2}_{1\,0}  + \frac{\rho_s \delta B_\parallel\vth}{2 c} \Gamma^{3}_{1\,1}\right) - \frac{\vth A_\parallel}{k_\parallel c} \zeta_1 \Gamma^{2}_{1\,0} \right\}
\\ \nonumber
+& \frac{k_\perp\rho_i n_{0s}}{2} \frac{c}{B_0} \bm{\eta}_s\times\mathbf{k}\cdot\mathbf{\hat{z}}\left\{ \frac{1}{k_\parallel} \zeta_0 \left( \phi \Gamma^{4}_{1\,0}+ \frac{\rho_s \delta B_\parallel \vth}{2 c} \Gamma^{5}_{1\,1} \right) - \frac{A_\parallel \vth}{k_\parallel c} \zeta_1 \Gamma^{4}_{1\,0}\right. \\
		\nonumber +& \left. \frac{1}{k_\parallel} \zeta_2 \left(\phi \Gamma^{2}_{1\,0} + \frac{\rho_s \delta B_\parallel \vth}{2 c} \Gamma^{3}_{2\,0} \right) - \frac{A_\parallel \vth}{k_\parallel c} \zeta_3\Gamma^{2}_{1\,0}\right\}
\end{align}

\section{The Linear Collisionless Dispersion Relation}
We can now combine all of these into a full collisionless dispersion relation for linear gyrokinetic waves. The quasi-neutrality condition becomes
\begin{align}
\nonumber
D_{1\,1} \phi + D_{1\,2} \left(-\frac{\omega A_\parallel}{k_\parallel c}\right) + D_{1\,3} \left(\frac{\delta B_\parallel T_{0i}}{q_i B_0}\right) = 0
\end{align}
With the definitions
\begin{align}
\nonumber
D_{1\,1} &= \sum\limits_s \frac{T_{0i}}{T_{0s}}\left\{1 + \xi_s \left( 1 - \frac{\omega^{*}_s}{\omega} \right) \Gamma^{1}_{0\,0} \zeta_0 -\xi_s \frac{\omega_{\eta\,s}}{\omega} \left\{ \zeta_0 \Gamma^{3}_{0\,0} + \zeta_2 \Gamma^{1}_{0\,0} \right\} \right\}
\\ \nonumber
D_{1\,2} &= \sum\limits_s \frac{T_{0i}}{T_{0s}} \left\{ \left(1 - \frac{\omega^{*}_s}{\omega}\right) \zeta_1 \Gamma^{1}_{0\,0} + \frac{\omega_{\eta\,s}}{\omega}\left\{ \zeta_1 \Gamma^{3}_{0\,0} + \zeta_3 \Gamma^{1}_{0\,0}\right\}\right\}
\\ \nonumber
D_{1\,3} &= \sum\limits_s \frac{q_s}{q_i} \left\{ \left( 1- \frac{\omega^{*}_s}{\omega}\right)\xi_s \zeta_0 \Gamma^{2}_{1\,0} + \xi_s \frac{\omega_{\eta\,s}}{\omega} \left\{ \zeta_0 \Gamma^{4}_{1\,0} + \zeta_2 \Gamma^{2}_{1\,0} \right\}  \right\}
\end{align}
Where we have used the drift frequencies
\begin{align}
\omega^{*}_s&= \frac{c T_{0s}}{q_s B_0} \bm{\kappa} \times \mathbf{k} \cdot \mathbf{\hat{z}}
\\
\omega_{\eta\,s} &= \frac{c T_{0s}}{q_s B_0} \bm{\eta} \times \mathbf{k} \cdot \mathbf{\hat{z}}
\end{align}

Introducing the ratio of the temperatures $\tau = T{0i}/T_{0e}$, we can define the following expressions in a generalisation of the notation in \cite{howes2006agb},
\begin{align}
\label{Adef}
A = 1 &+ \chi_i \xi_i Z(\xi_i) \Gamma^1_{0\,0}(\alpha_i) - \xi_i \frac{\omega_{\eta\,i}}{\omega} \left\{\zeta_0(\xi_i)\Gamma^{3}_{0\,0} + \zeta_2(\xi_i)\Gamma^1_{0\,0}(\alpha_i)\right\} \\ \nonumber
&+ \tau\left( 1 + \chi_e \xi_e Z(\xi_e) \Gamma^1_{0\,0}(\alpha_e) - \xi_e \frac{\omega_{\eta\,i}}{\omega} \left\{\zeta_0(\xi_e)\Gamma^{3}_{0\,0} + \zeta_2(\xi_e)\Gamma^1_{0\,0}(\alpha_e)\right\} \right)
\\
B &= 1 - \chi_i \Gamma^{1}_{0\,0}(\alpha_i) + \frac{\omega_{\eta\,i}}{\omega}\left\{ \Gamma^{3}_{0\,0}(\alpha_i)+\frac{1}{2}\Gamma^{1}_{0\,0}(\alpha_i) \right\} \\ \nonumber
&+ \tau\left(1 - \chi_e \Gamma^{1}_{0\,0}(\alpha_e) + \frac{\omega_{\eta\,i}}{\omega}\left\{ \Gamma^{3}_{0\,0}(\alpha_e)+\frac{1}{2}\Gamma^{1}_{0\,0}(\alpha_e) \right\}\right) 
\\
C &= \chi_i \xi_i \zeta_0(\xi_i) \Gamma^{2}_{1\,0}(\alpha_i) + \xi_i \frac{\omega_{\eta\,i}}{\omega}\left\{ \zeta_0(\xi_i) \Gamma^{4}_{1\,0}(\alpha_i) + \zeta_2(\xi_i) \Gamma^{2}_{1\,0}\right\} \\ \nonumber
&+ - \chi_e \xi_e \zeta_0(\xi_e) \Gamma^{2}_{1\,0}(\alpha_e) - \xi_e \frac{\omega_{\eta\,i}}{\omega}\left\{ \zeta_0(\xi_e) \Gamma^{4}_{1\,0}(\alpha_e) + \zeta_2(\xi_e) \Gamma^{2}_{1\,0}\right\} 
\end{align}
Where $\chi_s = 1 - \omega^{*}_s / \omega$. These simplify the $D_{i\,j}$ to $D_{1\,1} = A$, $D_{1\,2} = A - B$ , $D_{1\,3} = C$. Treating the parallel Amp\`ere's law similarly gives,
\begin{align}
D_{2\,1}\phi - D_{2\,2} \left(\frac{\omega A_\parallel}{k_\parallel c}\right) + D_{2\,3} \left(\frac{T_{0i} \delta B_\parallel}{q_i B_0}\right)  = 0
\end{align}
\begin{align}
\nonumber
D_{2\,1} &= A - B \\ \nonumber
D_{2\,2} &= A - B - \frac{\alpha_i v^2_{A}}{\omega^2} \\ \nonumber
D_{2\,3} &= C + E
\end{align}
\begin{align}
E &= \chi_i \Gamma^{2}_{1\,0}(\alpha_i) + \frac{\omega_{\eta_i}}{\omega}\left\{ \Gamma^{4}_{1\,0}(\alpha_i) + \frac{1}{2}\Gamma^{2}_{1\,0}(\alpha_i)\right\}
- \chi_e \Gamma^{2}_{1\,0}(\alpha_e) - \frac{\omega_{\eta_e}}{\omega}\left\{ \Gamma^{4}_{1\,0}(\alpha_e) + \frac{1}{2}\Gamma^{2}_{1\,0}(\alpha_e)\right\}
\end{align}
Where we use the manipulation
\[
\frac{k^2_\perp k^2_\parallel c^2}{4\pi \omega^2} \frac{T_{0i}}{q^2_i n_{0i}} = \frac{ \alpha_i v_A^2}{\omega^2}
\]
from \cite{howes2006agb} and denote by $v_A$ the Alfv\`en velocity ($B_0 / \sqrt{4\pi m_i n_{0i}}$).
Finally we manipulate the perpendicular Amp\`ere's law as in \cite{howes2006agb} to obtain 
\begin{align}
D_{3\,1}\phi - D_{3\,2} \left(\frac{\omega A_\parallel}{k_\parallel c}\right) + D_{3\,3} \left(\frac{T_{0i} \delta B_\parallel}{q_i B_0}\right)  = 0
\end{align}
\begin{align}
\nonumber
D_{3\,1} &= C \\ \nonumber
D_{3\,2} &= C + E \\ \nonumber
D_{3\,3} &= D - \frac{2}{\beta_i}
\end{align}
\begin{align}
D =&- \chi_i \xi_i \zeta_0(\xi_i)\Gamma^{3}_{0\,0}(\alpha_i) + \xi_i \frac{\omega_{\eta_i\,s}}{\omega} \left\{ \zeta_0(\xi_i) \Gamma^5_{1\,1}(\alpha_i) + \zeta_2(\xi_i) \Gamma^{3}_{1\,1}(\alpha_i) \right\} \\ \nonumber
&+ \chi_e \xi_e \zeta_0(\xi_e)\Gamma^{3}_{0\,0}(\alpha_e) - \xi_e \frac{\omega_{\eta_e\,s}}{\omega} \left\{ \zeta_0(\xi_e) \Gamma^5_{1\,1}(\alpha_e) + \zeta_2(\xi_e) \Gamma^{3}_{1\,1}(\alpha_e) \right\}
\end{align}
With this notation we arrive at the same dispersion relation as \cite{howes2006agb} , 
	  \begin{align}
	  \label{gkdisp}
	  \left( \frac{\alpha_i A}{\overline{\omega}^2} - AB + B^2\right)\left(\frac{2A}{\beta_i} - AD + C^2\right) = \left( AE + BC \right)^2
	  \end{align}
with the generalised definitions above. The definitions of our $D_{i\,j}$ also correspond (with a slight renumbering) to the shear free magnetic field limit of the standard low frequency linear dispersion relation in e.g. \cite{pu1985fbs}.
\bibliographystyle{apsrev4-1}
\bibliography{references.bib}
\end{document}

